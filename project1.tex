% project1.tex

\documentclass{article}
\usepackage{graphicx}
% Package useful for including figures
\title{P3800 Project 1: Numerical Differentiation and Integration}
%\author{Your Friendly Neighbourhood Professors}
\date{January 2019}
\begin{document}
   \maketitle


\section{Latex}

Just by getting to the point of being able to compile this document, you've come 
quite a way!  The project reports for this course are to be
typeset in Latex.  It's easy to make mistakes and things can get pretty
frustrating, especially in the beginning.  But it's good for you!
Most scientific writing for journal submissions (and books) is done using Latex.
And once you get the hang of it, writing out equations is much faster
than using other menu-driven typesetting programs.  You can quickly
find documentation on the web for Latex.  One useful website is

\begin{verbatim}
http://en.wikibooks.org/wiki/LaTeX
\end{verbatim}

\section{Makefiles}

It may be easy to keep track of compiling simple programs, e.g. with
\begin{verbatim}
gcc myprogram.c -o myprogram
\end{verbatim}

or

\begin{verbatim}
gfortran myprogram.f90 -o myprogram
\end{verbatim}
But once there are many source files using different libraries and the compilation
line gets a little busier, it helps to use makefiles to keep track of everything.
The makefile also keeps track of what needs to be updated so that not all the source
code needs to be compiled every time you make a change in one source file.

You can read more about the command {\bf make} and makefiles at
\begin{verbatim}
http://www.gnu.org/software/make/manual/make.html
\end{verbatim}

or by typing
\begin{verbatim}
man make
\end{verbatim}

Also, having a makefile makes it easier for others (like people marking your assignments) to 
compile your code.  For example, there
is a file called {\it Makefile} in this directory that governs how a  piece of code called 
{\it differentiate.f90} is compiled by the {\it make} utility.  Just type {\bf make differentiate} and the utility will use the contents of {\it Makefile}
to produce the executable {\it differentiate}.  To run the compiled program, just type {\bf ./differentiate}.  If all is well, running the code should produce a file called a file called {\it diff\_results.dat}.

\section{Numerical Differentiation}

\subsection{Differentiating a known function}

The program {\it differentiate.f90} calculates the derivative of 
$\sin(x)$ at $x=1$~rad, using a forward difference and
a central difference.  The program outputs the
absolute error $\left| {\rm numerical \,\, result} - {\rm exact} \right|$ of the calculations as a function of 
step size $h$ to a file called {\it diff\_results.dat}.  The results are graphed in Fig.~\ref{diff_results}.  The graphic 
was produced using a program called {\bf Grace}, freely available for Unix-like systems.
\begin{verbatim}
http://plasma-gate.weizmann.ac.il/Grace/
\end{verbatim}


\begin{figure}[h]
\includegraphics[width=\textwidth]{diff_results}
\caption{
Absolute error as a function of step size $h$ for forward and central difference method
in calculating the derivative of $\sin{x}$ at $x=1^r$.
Note the minimum in the error.  Also note the slopes of the curves on the high
side of the minimum.  The $O(h)$ method (forward difference, circles) has slope
one in logarthmic units, while the
$O(h^2)$ method (central difference, squares) has slope two.
}
\label{diff_results}
\end{figure}

You can recreate most of the the plot by typing

\begin{verbatim}
xmgrace -log xy -nxy diff_results.dat
\end{verbatim}


Note that depending on how the program was installed, it could be called  {\tt xmgr}.
The {\tt -nxy} tells {\bf Grace} to use the first column in the data file as the $x$ values (horizontal axis),
and all the other columns as different sets of $y$ values.  Otherwise, only the first two columns are
plotted (the first on the horizontal, the second on the vertical).  The {\tt -log xy} tells {\bf Grace} to use logarthmic scaling for the $x$ and $y$ axes. 
By double clicking on the graph in the {\bf Grace} window, you will get a menu of options for 
plotting symbols.  By double clicking near an axis, you can add a title to that axis, play with tick marks
and labels, etc.  You can also
access these features through the various menus.  To generate an {\tt eps} (encapsulated postscript) file
used by Latex for making figures, select {\tt Print setup} from the {\tt File} menu.  There, choose {\tt eps}
from the {\tt Device} list and choose an appropriate name for the file (default is usually okay).  Then, when you
print ({\tt Print} from the {\tt File} menu), an {\tt eps} file will be generated.
\\




{\bf TASK \#1:}  Modify {\it differentiate.f} so that it calculates the derivative of 
$\sin(x)$ at $x=1$~rad using the fourth order method
shown in class
$$
f^\prime(x) = \frac{1}{12h} \left( f[x-2h] - 8f[x-h] + 8f[x+h] - f[x+2h] \right) + \mathcal{O}(h^4).
$$
%, equation 5.10 on page 39, noting the comments on page 40.  
Plot the results together with those already shown in Fig.~\ref{diff_results}.  
In the main body of your report, {\bf briefly} discuss your results. In particular, comment on the minimum errors
attained for the different methods, and the values of $h$ at which they occur.   Do the results match 
theoretical estimates obtained in class?

\subsection{Differentiating data}

In the project directory, you should also find a file called {\it experiment.dat}.  Column 1 is the temperature $T$
(on the Kelvin scale), while column 2 reports a dimensionless quantity $s$ related to the sound velocity
in the compound UNi$_2$Si$_2$.  Phase transitions are marked by sudden changes in the derivative of 
$s$ with respect to $T$.  The data are not equally spaced in $T$.

In general, for a function $f(x)$ sampled at a set of (unequally) spaced points, the derivative is
$$
f'_i=\frac{h^2_{i-1} f_{i+1}+(h^2_i-h^2_{i-1})f_i-h^2_i f_{i-1}}{h_i h_{i-1} (h_i+h_{i-1})}
+O(h^2),
$$
where $f_i=f(x_i)$, $h_i=x_{i+1}-x_i$ and $h$ is the larger of $h_i$ and $h_{i-1}$.
\\

{\bf TASK \#2}
Write a program that uses the above formula to calculate the slope $\frac{d s}{dT}$.  Plot your 
result.  Ignoring the initial noise at $T$ near zero, how many phase transitions do you see? 

\section{Numerical Integration}

In the study of blackbody radiation, there is a result, Stefan's Law, that states that the power
emitted from a perfect blackbody per unit area is
$$
j=\sigma T^4,
$$
where $T$ is the absolute temperature and $\sigma$ is a constant that can be expressed as,
$$
\sigma = \frac{2 \pi k^4}{c^2 h^3} \int_0^\infty {\rm d}u \frac{u^3}{e^u -1}.
$$
It turns out that the integral in the above expression can be done analytically, and has a value of $\pi^4/15$.
\\

{\bf TASK \#3:}
As an exercise in numerical integration, calculate the integral
$$
\int_0^\infty {\rm d}u \frac{u^3}{e^u -1},
$$
using Gauss-Laguerre integration (see handout from Klein and Gordan).
The parameters for $N=2$, 4, 8, 16 \dots 32 are provided in the
files labelled {\it weights.dat\_N}.  These files were obtained from the website quoted in the book.
Check the website to see what the columns of the files are.
Report the absolute value of the relative error, 
(exact - num val)/exact,
as a function of $N$ in a table.  Look at Table~\ref{tab1} as a template.  
Also, make a plot of the value of the integral as a function of the order.  {\bf Briefly} discuss the results.  
Note that the parameters are given to 12 digits of precision, while a double precision real 
number generally has 14 decimal digits of precision.

The program {\it integrate.f90} provides a start. 
Compile the code by typing 
%begin{verbatim}
{\bf make integrate}.
%\end{verbatim}
Examine the program source
code as well as the script {\it run\_integrate.sh} to figure out how you might obtain 
the required table of values.  The shell script {\it run\_integrate2.sh} is an alternate 
script that accomplishes the same task but in a slightly different way.  To run a script, just type
{\bf ./run\_integrate2.sh}.  When writing your own script, make sure to change permissions to 
make it executable.
\\

{\bf TASK \#4:}
Write your own program to evaluate the integral using Simpson's rule.  
When writing your code, define a separate function for your integrand.  This will enable you and others to 
easily modify your code to integrate some other function.
One difficulty in using Simpson's rule is that you can not
integrate out to infinity, but rather must choose an
upper limit to the integral.  The value of the integral should be insensitive to
within desired precision to the upper limit of integration, i.e., once the upper limit is large enough,
making it larger won't change the value of your answer.  Part of your task, therefore, is to explore
how using Simpson's rule for the integral at hand depends on step size and this upper limit of integration.

Concretely, carry out the integral 
$$
\int_0^x  {\rm d}u \frac{u^3}{e^u -1},
$$
using $n=11$, 101, 1001, $10^4+1$, $10^5+1$, $10^6+1$, and $10^7+1$ points at which to evaluate the integrand 
(remembering that Simpson's rule requires an odd number of points), for $x=10$, 100, 1000 and 10000.
Your program, or program/script combination, should produce four data files, each containing absolute relative error as a function of
$n$ for a given $x$.  Using {\bf grace}, plot the four curves use logarithmic axes.
{\bf Briefly} discuss your results and compare them to 
the Gauss-Laguerre method.
\\ 

%{\bf TASK \#5:}
%You should be curious as to how {\it Mathematica} handles the integral, both 
%analytically and numerically.  Try it, and briefly report on what you get.

\begin{table}
\begin{tabular}{|c|c|c|}
\hline
Order & value of integral & relative error \\
\hline
2 &  818 &  h\\
4 &  2.319556 & e \\
8 & 704 & l\\
16 & 46 & l\\
20 & 0 &  o \\
24 & 0 & o \\
28 & 0 & o \\
32 & 35 & o\\
\hline
\end{tabular}
\caption{
Sample table.  Use this table as a template to report the values of the integral to be done using
Gauss-Laguerre integration, as well as the absolute relative error.}
\label{tab1}
\end{table}

\section{YOUR SUBMISSION}

You need to submit a hardcopy of your report that includes graphs, tables and discussion related to 
each task.  All tasks carry the same weight for marking purposes.

Please submit all relevant source code, scripts, latex, eps and pdf files electronically as shown in class. We should be able to simply type {\bf latex project1.tex} to reproduce your report.  {\bf Please write and  
include a script called {\it run\_project1.sh}, which when run should compile and run code and output any relevant data to the screen. } Use the {\it run\_project1.sh} provided as a template if you wish.  The idea is that by running {\it run\_project1.sh}, we should quickly see that all your code compiles, that it produces meaningful output, and whether your numbers are right.  


\end{document}
